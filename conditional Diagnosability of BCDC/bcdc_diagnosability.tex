\documentclass[11pt,a4paper]{article}

\usepackage{latexsym}
\usepackage{pict2e}
\usepackage{setspace}
\usepackage{amsmath,amsthm,amssymb}
\usepackage{graphics}
\usepackage{graphicx}
\usepackage{indentfirst}
\usepackage{epstopdf}
\usepackage{epsfig}
\usepackage{caption}
\usepackage{setspace}
\usepackage{xcolor}
\topmargin 0cm \textheight 22.6cm \textwidth 17cm \oddsidemargin
-0.6cm \evensidemargin -0.8cm
\parskip 4pt

\newtheorem{definition}{Definition}

\renewcommand{\thefootnote}{\fnsymbol{footnote}}

\usepackage{amsmath, amsthm}
{
\renewcommand{\thefootnote}{\fnsymbol{footnote}}
\newtheorem{thm}{Theorem}
\newtheorem{lem}{Lemma}
\newtheorem{cor}{Corollary}
\newtheorem{Def}{Definition}
\newtheorem{claim}{Claim}
\newtheorem{prop}{Proposition}
\newtheorem{conjecture}{Conjecture}
\newtheorem{notation}{Notation}
\newtheorem{remark}{Remark}
\newtheorem{example}{Example}
}



\usepackage{graphicx}
\graphicspath{{fig/}}

\title{The Conditional Diagnosability of BCDC under the Comparison Diagnosis Model}

\author{Shuangxiang Kan, Baolei Cheng, and Jianxi Fan \footnote{Corresponding author: jxfan@suda.edu.cn (J. Fan)}\\
School of Computer Science and Technology, Soochow University,
Suzhou, China\\
}

\date{}

\begin{document}
\maketitle
%\captionsetup[figure]{labelfont={bf},name={Fig.},labelsep=period}
%\begin{center}





\begin{abstract}
abstract
\end{abstract}

\noindent{\bf Keywords:}  BCDC, conditional diagnosability

\doublespacing

\section{Introduction}
introducion

\section{Notations and Definitions}\label{notations}

\begin{definition}
\label{pair-related-CQn}
\cite{wang2018bcdc}
Two binary strings $x = x_1x_0$ and $y = y_1y_0$ of length $2$ are said to be pair-related
$($denoted by $x \sim y$$)$ if and only if $(x,y) \in \{(00, 00), (10, 10), (01, 11), (11, 01)\}.$
\end{definition}


\textcolor{red}{$E(R_0,R_2)$}

\textcolor{red}{$B_n-F-V(A)$}

\textcolor{red}{$S_n \setminus F_2 \neq \emptyset$}

\textcolor{red}{$|F_0|$}


\begin{definition}
\label{Defi-CQn}
\cite{wang2018bcdc}
The $n$-dimensional crossed cube$,$ $CQ_n$$,$
is recursively defined as follows$.$
$CQ_1$ is the complete $($undirected$)$ graph on two nodes whose addresses are $0$ and $1.$
$CQ_n$ consists of $CQ^0_{n-1}$ and $CQ^1_{n-1}.$
The most significant bits of the addresses of the nodes in $CQ^0_{n-1}$ and $CQ^1_{n-1}$ are $0$ and $1,$  respectively.
The nodes $u = u_{n-1}u_{n-2}\cdots u_0 \in V(CQ^0_{n-1})$
and $v = v_{n-1}v_{n-2}\cdots v_0 \in V(CQ^1_{n-1}),$
where $u_{n-1} = 0$ and $v_{n-1} = 1,$  are joined by an edge in $CQ_n$ if and only if

$1)$ $u_{n-2} = v_{n-2}$ if $n$ is even$,$  and

$2)$ $u_{2i+1}u_{2i} \sim v_{2i+1}v_{2i}$ $($see Definition $\ref{pair-related-CQn})$$,$
for $\lfloor \frac{n - 1}{2} \rfloor > i\geq 0.$
\end{definition}

\begin{definition}
\label{Defi-BCDCn}
\cite{wang2018bcdc}
The $n$-dimensional BCDC network$,$
$B_n,$
is recursively defined as follows$.$
$B_2$ is a cycle with $\rm{4}$ nodes
$[00,01],$
$[00,10],$
$[01,11],$ and
$[10,11].$
For $n \geq 3,$
we use $B^0_{n-1}$ $($resp$.$ $B^1_{n-1}$$)$ to denote
the graph obtained by $B_{n-1}$ with changing each node $[x,y]$ of $B_{n-1}$ to $[0x,0y]$ $($resp$.$ $[1x,1y]$$).$
$B_n$ consists of $B^0_{n-1}$, $B^1_{n-1},$  and a node set
$S_n = \{ [a,b] | a \in V(CQ_{n - 1}^0),$ $b \in V(CQ_{n - 1}^1),$ and $(a,b) \in E(C{Q_n})\}$
according to the following rules$.$
For nodes  $u = [a,b] \in V(B_{n - 1}^0),$ $v = [c,d] \in S_n,$ and $w = [e,f] \in V(B_{n - 1}^1)$$:$

$1)$  $(u,v) \in E(B_n)$ if and only if $a = c$ or $b = c.$

$2)$  $(v,w) \in E(B_n)$ if and only if $e = d$ or $f = d.$
\end{definition}

\begin{lem} \label{distinguishable} \rm
Let $G=(V,E)$ be a system. For any two distinct subsets $F_1$ and $F_2$ of $V(G)$, $(F_1,F_2)$ is a distinguishable pair if and only if at least one of the following conditions is satisfied:

%�����C��û�ж���
1)$\exists u,v \in F_1 \setminus F_2$ and $\exists w \in V(G) \setminus (F_1 \cup F_2)$ such that $(u,v)_w \in C$.

2)$\exists u,v \in F_2 \setminus F_1$ and $\exists w \in V(G) \setminus (F_1 \cup F_2)$ such that $(u,v)_w \in C$.

3)$\exists u,w \in V(G) \setminus (F_1 \cup F_2)$ and $\exists v \in F_1 \bigtriangleup F_2$ such that $(u,v)_w \in C$.
\end{lem}


We use $cn(G)$ to represent the maximum number of common nodes between any two nodes in graph $G$.

\begin{lem} \label{connectivity} \rm
For an $n-$dimensional BCDC, $B_n$ has the following properties:

(1) $B_n$ has $n2^{n-1}$ nodes and $n(n-1)2^{n-1}$ edges.

(1) $B_n$ is $(2n-2)-$regular, $\kappa(B_n)=2n-2$.
\end{lem}





\begin{lem} \label{cn} \rm
$cn(B_n,x,y)=n-2$ if $(x,y)\in E(B_n)$ and $cn(B_n,x,y)=2$ or 0 if if $(x,y)\notin E(B_n)$ for $n>=3$.
\end{lem}

\begin{lem}
\label{nodesNeiinBn}
\cite{wang2018bcdc}
For any integer  $n \geq 3$ and any two nodes $u=[a,b],v=[c,d] \in S_n,$  we have
\[
\footnotesize
|N_{B_{n-1}^0}(\{u,v\}) | =
\begin{cases}
  2n-2,   &  \textrm{if} \ (a,c) \notin E(CQ_n),\\
  2n-3,   &  \textrm{if} \ (a,c) \in E(CQ_n),\\
\end{cases}
\]
and
\[
\footnotesize
|N_{B_{n-1}^1}(\{u,v\}) | =
\begin{cases}
  2n-2,   &  \textrm{if} \ (b,d) \notin E(CQ_n),\\
  2n-3,   &  \textrm{if} \ (b,d) \in E(CQ_n).\\
\end{cases}
\]
\end{lem}

\begin{lem} \label{SnNeighboursNode} \rm
For any three nodes $x$, $y$ and $z$ in $B^0_{n-1}$ where $(x,y),(y,z) \in E(B_n)$ and $n \geq 3$, $|N_{S_n}(N_{B^0_{n-1}}(x,y,z) \cup \{x,y,z\})| > |N_{B^0_{n-1}}(x,y,z)|$.
\end{lem}

\begin{proof}
Let $x=[a,b]$, $y=[c,d]$ and $z=[e,f]$ and $H=\{a,b\} \cup \{c,d\} \cup \{e,f\}$. Since $(x,y),(y,z) \in E(B_n)$ and $CO_n$ contains no circle of length 3, $|H| = 4$. Without loss of generality, suppose that $b=c$, $d=e$, If $(a,f) \in E(CQ^0_{n-1})$ or there exists a node $m$ in $CQ^0_{n-1}$, which $(a,m), (m,f) \in E(CQ_{n-1})$, then  $|N_{B^0_{n-1}}(x,y,z)|= |N_{CQ^0_{n-1}}(a,b,d,f)|+1$. For each node in $N_{CQ^0_{n-1}}(H) \cup H$, there is a matching node in $CQ^1_{n-1}$. According to the construction method of BCDC, we have $|N_{CQ^0_{n-1}}(H) \cup H|= |N_{S_n}(N_{B^0_{n-1}}(x,y,z)) \cup \{x,y,z\}| > |N_{B^0_{n-1}}(u)|$.
\end{proof}


\section{The Conditional Diagnosability of BCDC under the Comparison Diagnosis Model }\label{MM}

\begin{lem} \label{F1belongstoF2} \rm
Let $G$ be a graph \textcolor{red}{$\delta(G)>2$}, and let $F_1$ and $F_2$ be any two distinct conditional faulty sets of $V(G)$ with $F_1 \subset F_2$. Then, $(F_1,F_2)$ is a distinguishable conditional pair under the comparison diagnosis model.
\end{lem}

\begin{lem} \label{conditionalDiagnosability<=} \rm
$t_c(B_n) \leq 4n-6$, for $n \geq 6$.
\end{lem}

\begin{proof}
Consider a path $P=\langle u,v,w \rangle$ where $u[0]=v[0]=w[0]$. Let $|F_1|=N_{B_n}(P) \cup \{u\}$ and $|F_2|=N_{B_n}(P) \cup \{w\}$. Since $|F_1 \setminus F_2|=|F_2 \setminus F_1|=1$, condition 1 and condition 2 of Lemma \ref{distinguishable} are not satisfied; $F_1 \bigtriangleup F_2=\{u,w\}$, for each node $x \in F_1 \bigtriangleup F_2$, there is only one node $v \in V(G) \setminus (F_1 \cup F_2)$ which is adjacent to $x$ and $v$ does not have a neighbour in $V(G) \setminus (F_1 \cup F_2)$, then condition 3 of Lemma \ref{distinguishable} is not satisfied. Thus $F_1$ and $F_2$ are indistinguishable. By Lemma \ref{cn} and $\{(u,v),(u,w),(v,w)\} \subset E(B_n)$, $|F_1 \cap F_2|=|N_{B_n}(P)|=4n-6$ and $|F_2 \setminus F_1|=|F_1 \setminus F_2|=1$. Hence, $|F_1|=|F_2|=4n-5$. Next, we show that $F_1$ and $F_2$ are conditional faulty sets. Without loss of generality, for each node $x \in V(B_n)$, we show $N_{B_n}(x) \subseteq F_1$, another situation $N_{B_n}{x} \subseteq F_2$ is similar.

{\bf Case 1.} $x \in V(P)$. If $x=u$, then $x$ has a neighbour $v \notin F_1$; If $x=W$, then $x$ has a neighbour $v \notin F_1$; If $x=v$, then $x$ has a neighbour $w \notin F_1$.

{\bf Case 2.} $x \in N_{B_n}(P)$. If $x \in N_{B_n}(v)$, then $x$ has a neighbour $v \notin F_1$; If $x \in N_{B_n}(w)$, then $x$ has a neighbour $w \notin F_1$; If $x \in N_{B_n}(u)$, by Definition \ref{Defi-BCDCn}, $x$ has two neighbours in $S_n$ and $u$ has two neighbours in $S_n$. Since $N_{S_n}(x) \cap N_{S_n}(u) \leq 1$, then $x$ has a neighbour which does not belong to $F_1$;

{\bf Case 3.} $x \notin N_{B_n}(P) \cup \{u,v,w\}$. $x$ is not adjacent to any node in $P$, by Lemma \ref{cn}, max$(N_{B_n}(x,y))=2$ for each node $y$ in $P$. Since $\kappa(x)=2n-2 > 6$ for $n \geq 5$, $x$ has a neighbour which does not belong to $F_1$;

Therefore, $t_c(B_n) \leq 4n-6$, for $n \geq 6$.
\end{proof}

\begin{lem} \label{BnK1} \rm
Let $F$ be a faulty set of $B_n$ with $|F| \leq 3n-5$,
where $n \geq 3$. Then, $B_n-F$ satisfies one of the following
two conditions: 1) $B_n-F$ is connected. 2) $B_n-F$ has
two components, one is $K_1$ and the other contains $n2^{n-1}-|F|-1$ nodes.
\end{lem}

\begin{proof}
We prove the claim by induction on $n$. It is easy to verify that this Lemma holds when $n=3$, so our proof starts from $n \geq 4$. Suppose that $m_1$ and $m_2$ are two $K_1$s in $B_n-F$. According to the Lemma \ref{connectivity}, we have $|N(m_1)|=|N(m_2)|=2n-2$ and $|N(m_1) \cap N(m_2)| \leq 2$. Then $|S| \geq |V(m_1) \cup V(m_2)|=|V(m_1)|+|V(m_2)|-|N(m_1) \cap N(m_2)| \geq (2n-2)+(2n-2)-2=4n-6 > 3n-5$ for $n \geq 4$. Hence, there exists at most one $K_1$ in $B_n-F$. Let $A$ be a $K_1$ in $B_n-F$, then we will show that $R=B_n-F-V(A)$ is connected. Let $F_0=F \cap V(B^0_{n-1})$, $F_1=F \cap V(B^1_{n-1})$, $F_2 = F \cap V(S_n)$.
%$R_0=B[V(R) \cap V(B^0_{n-1})]$, $R_1=B[V(R) \cap V(B^1_{n-1})]$, and $R_2 = B[V(R) \cap S_n]$.
Then we deal with the following cases.

{\bf Case 1.} $F=F_0$ or $F=F_1$. If $F=F_0$, then $F_1=F_2=\emptyset$. We can easily verify that $B[V(B^1_{n-1}) \cup S_n]$ is connected. For each node $x$ in $ V(B^1_{n-1}) \setminus F_0$, by Definition \ref{Defi-BCDCn}, there are two nodes in $S_n$ which are adjacent to $x$. Thus, $B_n-F-V(A)$ is connected; If $F=F_1$, this case is the same as $F=F_0$.

{\bf Case 2.} $F=F_2$. If $F=F_2$, then $F_0=F_1=\emptyset$. We can easily verify that $B^0_{n-1}$ and $B^1_{n-1}$ are connected, respectively. For $n \geq 4$, $|S_n|=2^{n-1} > 3n-5$, then $S_n \setminus F_2 \neq \emptyset$. For each node $u$ in $S_n \setminus F_2$, by Definition \ref{Defi-BCDCn}, $u$ has $n-1$ neighbours in $R_0$ and $n-1$ neighbours in $R_1$, respectively. Thus, $B_n-F-V(A)$ is connected.

{\bf Case 3.} $F_0 \neq \emptyset$, $F_1 \neq \emptyset$, and $F_2 = \emptyset$. Then, we have the following two subcases.

{\bf Subcase 3.1.} $n \leq |F_0| \leq 3n-6$ and $1 \leq |F_1| \leq 2n-5$. Since each node $u$ in $S_n$ has $(n-1)$ neighbours in $V(B^1_{n-1})$ and by Lemma \ref{nodesNeiinBn}, there exists at most a node $u_1 \in S_n$, where $N_{B^1_{n-1}}(u_1) \subset F_1$. For each node $u$ in $S_n \setminus u_1$, there are $(n-1)$ nodes in $V(B^1_{n-1})$ which are adjacent to $u$ and $R_1$ is connected due to $\kappa(B_{n-1})=2n-4$. Thus, $B_n[\{S_n \setminus{u_1}\} \cup V(B^1_{n-1})]$ is connected. Since $B_n-F-V(A)$ does not contain a $K_1$, there exists a node $x_1$ in $V(B^1_{n-1}) \setminus F_0$ where $(x_1, u_1)\in E(B_n)$. For each node $x$ in $V(B^0_{n-1})\setminus F_0$, there are two nodes in $S_n$ which are adjacent to $x$. Then, $B_n-F-V(A)$ is connected.

{\bf Subcase 3.2.} $n \leq |F_1| \leq 3n-6$ and $1 \leq |F_0| \leq 2n-5$. This argument is similar to Subcase 3.1.

{\bf Case 4.} $F_0 \neq \emptyset$, $F_2 \neq \emptyset$, and $F_1 = \emptyset$. Then, we have the following three subcases.


{\bf Subcase 4.1.} $1 \leq |F_0| \leq 2n-5$ and $n \leq |F_2| \leq 3n-6$. Since $\kappa(B_{n-1})=2n-4$ and $F_1 = \emptyset$, $B_n[V(B^0_{n-1})\setminus F_0]$ and $B^1_{n-1}$ are connected, respectively. For $n \geq 4$, $|S_n|=2^{n-1} > 3n-6$, then $S_n \setminus F_2 \neq \emptyset$. For each node $u$ in $S_n \setminus F_2$, by Definition \ref{Defi-BCDCn}, $u$ has $n-1$ neighbours in $V(B^1_{n-1})$. Thus, $B_n[\{S_n\setminus F_2\} \cup V(B^1_{n-1})]$ is connected. For each node $x\in V(B^0_{n-1})$, there are two nodes in $S_n$ which are adjacent to $x$ and for each node $u\in S_n$, $u$ has $n-1$ neighbours in $V(B^0_{n-1})$. By Definition \ref{Defi-BCDCn}, \textcolor{red}{
$|E(R_0,R_2)| \geq (n-1)2^{n-1}-n(3n-6)-2 > 0$ for $n \geq 4$}. Thus, there exist two nodes $x \in V(B^0_{n-1})\setminus F_0$, $u \in S_n$ and $(x,u) \in E(B_n)$. Since $B_n[V(B^0_{n-1})\setminus F_0]$ and $B_n[\{S_n\setminus F_2\} \cup V(B^1_{n-1})]$ are connected, respectively, $B_n-F-V(A)$ is connected.

{\bf Subcase 4.2.} $2n-4 \leq |F_0| \leq 3n-7$ and $2 \leq |F_2| \leq n-1$. We can easily verify that $B_n[\{S_n\setminus F_2\} \cup V(B^1_{n-1})]$ is connected. If $B_n[V(B^0_{n-1})\setminus F_0]$ is connected, similar to Subcase 4.1, $B_n-F-V(A)$ is connected; Suppose that $R_0$ is disconnected, then $R_0$ consists of two components, one is $K_1$ and the other contains $|V(R_0)|-1$ nodes. Let $x_1$ be the $K_1$ and $L$ be another component in $R_0$. Since there is only one $K_1$ in $B_n-F$, then $x_1$ has a neighbour in $R_2$. Thus $B_n[\{S_n\setminus F_2\} \cup V(B^1_{n-1}) \cup \{x_1\}]$ is connected. By Definition \ref{Defi-BCDCn},
$|E(L,R_2)| \geq (n-1)2^{n-1}-n(n-1)-2(2n-4) > 0$ for $n \geq 4$. Thus, there exist two nodes $x \in V(L)$, $u \in V(S_n\setminus F_2)$ and $(x,u) \in E(B_n)$. Since $L$ and $B_n[\{S_n\setminus F_2\} \cup V(B^1_{n-1}) \cup \{x_1\}]$ are connected, respectively, $B_n-F-V(A)$ is connected.

{\bf Subcase 4.3.} $|F_0| = 3n-6$ and $|F_2| = 1$. We can easily verify that $B_n[\{S_n\setminus F_2\} \cup V(B^1_{n-1})]$ is connected. For each node $x$ in $ V(B^0_{n-1}\setminus F_0)$, by Definition \ref{Defi-BCDCn}, there are two nodes in $S_n$ which are adjacent to $x$. Thus, $B_n-F-V(A)$ is connected;

{\bf Case 5.} $F_1 \neq \emptyset$, $F_2 \neq \emptyset$, and $F_0 = \emptyset$. This argument is similar to Subcase 3.1.

{\bf Case 6.} $F_0 \neq \emptyset$, $F_2 \neq \emptyset$, and $F_1 \neq \emptyset$. Then, we have the following four subcases.

{\bf Case 6.1.} $1 \leq |F_0| \leq 2n-5$, $1 \leq  |F_1| \leq 2n-5$, and $1 \leq |F_2| \leq 3n-7$. Since $\kappa(B_{n-1})=2n-4$, $B_n[B^0_{n-1} \setminus F_0]$, $B_n[B^1_{n-1} \setminus F_1]$ are connected respectively. By Lemma \ref{nodesNeiinBn}, there exists at most one node $u_1 \in S_n\setminus F_2$, where $N_{B^0_{n-1}}(u_1) \subset F_0$. For each $u \in S_n\setminus F_2 \setminus u_1$, $u$ has $(n-1)$ neighbours in $B^0_{n-1}$. Thus, $B[\{R_2 \setminus u_1\} \cap R_0]$ is connected. Similarly, there exists at most one node $u_2 \in S_n$, where $N_{B^1_{n-1}}(u_2) \subset F_1$ and $u_1 \neq u_2$, and $B[\{R_2 \setminus u_2\} \cap R_1]$ is connected. For $n \geq 4$, $|S_n|=2^{n-1}-(3n-7) > 2$, Thus, $B_n-F-V(A)$ is connected.

{\bf Case 6.2.} $2n-4 \leq |F_0| \leq 3n-7$, $1 \leq |F_1| \leq n-2$, and $1 \leq |F_2| \leq n-2$. We can easily verify that $B[V(R_1) \cup V(R_2)]$ is connected. If $R_0$ is connected, similar to Subcase 6.1, $B_n-F-V(A)$ is connected; Suppose that $R_0$ is disconnected, then $R_0$ consists of two components, one is $K_1$ and the other contains $V(R_0)-1$ nodes. Let $x_1$ be the $K_1$ and $L$ be another component in $R_0$. Since there is only one $K_1$ in $B_n-F$, then $x_1$ has a neighbour in $R_0$. Thus $B[V(R_1) \cup V(R_2) \cup \{x_1\}]$ is connected. By Definition \ref{Defi-BCDCn},
$|E(L,R_2)| \geq (n-1)2^{n-1}-n(n-2)-2(2n-4) > 0$ for $n \geq 4$. Thus, there exist two nodes $x \in V(L)$, $u \in V(R_2)$ and $(x,u) \in E(B_n)$. Since $L$ and $B[V(R_1) \cup V(R_2) \cup \{x_1\}]$ are connected, respectively, $B_n-F-V(A)$ is connected.

{\bf Case 6.3.} $2n-4 \leq |F_1| \leq 3n-7$, $1 \leq |F_0| \leq n-2$, and $1 \leq |F_2| \leq n-2$. This argument is similar to Subcase 6.2.

The above arguments indicate that if there is no $K_1$ in $B_n-F$, then $B_n-F$ is connected. Therefore, the Lemma holds.
\end{proof}



\begin{lem} \label{BnK2} \rm
Let $F$ be a conditional faulty set of $B_n$ with $|F| \leq 4n-7$,
where $n \geq 5$. Then, $B_n-F$ satisfies one of the following
two conditions: 1) $B_n-F$ is connected. 2) $B_n-F$ has
two components, one is $K_2$ and the other contains $n2^{n-1}-|F|-2$ nodes.
\end{lem}

\begin{proof}
We first prove the second case. Since $F$ is a conditional faulty set, there is no an isolated node in $B_n-F$. Suppose that $m_1$ and $m_2$ are two $K_2$s in $B_n-F$. According to the Lemma \ref{cn}, we have $|V(m_1)|=|V(m_2)|=(2n-3)+(2n-3)-(n-2)=3n-4$ and $|N(m_1) \cap N(m_2)| \leq 8$. Then $|S| \geq |V(m_1) \cup V(m_2)|=|V(m_1)|+|V(m_2)|-|N(m_1) \cap N(m_2)| \geq (3n-4)+(3n-4)-8=6n-16 > 4n-7$ for $n \geq 5$. Hence, there exists at most one $K_2$ in $B_n-F$. Let $A$ be a $K_2$ in $B_n-F$, then we will show that $R=B_n-F-V(A)$ is connected. Let $F_0=F \cap V(B^0_{n-1})$, $F_1=F \cap V(B^1_{n-1})$, $F_2 = F \cap V(S_n)$, $R_0=B[V(R) \cap V(B^0_{n-1})]$, $R_1=B[V(R) \cap V(B^1_{n-1})]$, and $R_2 = B[V(R) \cap V(S_n)]$. Then we deal with the following cases.

{\bf Case 1.} $F=F_0$ or $F=F_1$. If $F=F_0$, then $F_1=F_2=\emptyset$. We can easily verify that $B[V(R_1) \cup V(R_2)]$ is connected. For each node $x$ in $ V(R_0)$, by Definition \ref{Defi-BCDCn}, there are two nodes in $S_n$ which are adjacent to $x$. Thus, $B_n-F-V(A)$ is connected; If $F=F_1$, this argument is similar to $F=F_0$.

{\bf Case 2.} $F=F_2$. If $F=F_2$, then $F_0=F_1=\emptyset$. We can easily verify that $R_0$ and $R_1$ are connected, respectively. For $n \geq 5$, $|S_n|=2^{n-1} > 4n-7$, then $S_n \setminus F_2 \neq \emptyset$. For each node $u$ in $R_2$, by Definition \ref{Defi-BCDCn}, $u$ has $n-1$ neighbours in $R_0$ and $n-1$ neighbours in $R_1$, respectively. Thus, $B_n-F-V(A)$ is connected.

{\bf Case 3.} $F_0 \neq \emptyset$, $F_1 \neq \emptyset$, and $F_2 = \emptyset$. Then, we have the following four subcases.

{\bf Subcase 3.1.} $2n-2 \leq |F_0| \leq 4n-8$ and $1 \leq |F_1| \leq 2n-5$. By Lemma \ref{nodesNeiinBn}, there exists at most a node $u_1 \in S_n$, where $N_{R_1}(u_1) \subset F_1$. Since $\kappa(B_{n-1})=2n-4$, $R_1$ is connected. For each node $u$ in $R_2$, there are $(n-1)$ nodes in $V(B^1_{n-1})$ which are adjacent to $u$. Thus, $B[\{R_2\setminus{u_1}\} \cup V(R_1)]$ is connected. Since $F$ is a conditional faulty set, there exists a node $x$ in $V(R_0)$ where $(u_1,x)\in E(B_n)$. For each node $x_1$ in $V(R_0)$, there are two nodes in $S_n$ which are adjacent to $x_1$. Then, $B_n-F-V(A)$ is connected.

%{\bf Subcase 3.1.} $|F_0| \geq 2n-2$ and $|F_1| \leq 2n-5$. Since $\kappa(B_{n-1})=2n-4$, $R_1$ is connected. For any node $u \in S_n$, if $N_{R_1}(u) \subset F_1$ (respectively, $N_{R_0}(u) \subset F_0$), since $F$ is a conditional faulty set, there exists a node $x$ (respectively, $y$) in $V(R_0)$ (respectively, $V(R_1)$), where $(u,x)\in E(B_n)$ ($(u,y)\in E(B_n)$). For each node $x_1$ (respectively, $y$) in $V(R_0)$ (respectively, $V(R_1)$), there are two nodes in $S_n$ which are adjacent to $x_1$ (respectively, $y$). Then, $B_n-F-V(A)$ is connected. If $R_0$ is connected, similar to Subcase 3.1, $B_n-F-V(A)$ is connected; Suppose that $R_0$ is disconnected, for each node $x$ in $V(R_0)$, there are two nodes in $S_n$ which are adjacent to $x$. Thus, $B_n-F-V(A)$ is connected.

{\bf Subcase 3.2.} $|F_1| \geq 2n-2$ and $|F_0| \leq 2n-5$. This argument is similar to Subcase 3.1.

{\bf Subcase 3.3.} $|F_0| = 2n-3$ and $|F_1| = 2n-4$. If $R_1$ is connected, the argument is similar to Subcase 3.1; Suppose that $R_1$ is disconnected, by Lemma \ref{BnK1}, there exists only one isolated node $y_1$ and another connected component $L$. By Definition \ref{Defi-BCDCn}, $y_1$ has two neighbours $u_1$ and $u_2$ in $S_n$. By Lemma \ref{nodesNeiinBn}, min$(|N_{B_{n-1}^1}(\{u_1,u_2\})|)=2n-3 > N_{B_{n-1}^1(y_1)}=|F_1|=2n-4$, so there exists a node $y_2$ in $L$ which is adjacent to $u_1$ or $u_2$. Without loss of generality, we set $(y_2,u_1)\in E(B_n)$. Since $L$ is connected and $B[\{y_1,u_1,u_2\}]$ is connected, $B[S_n \cap V(R_1)]$ is connected. For each node $x$ in $V(R_0)$, there are two nodes in $S_n$ which are adjacent to $x$. Thus, $B_n-F-V(A)$ is connected.

{\bf Subcase 3.4.} $|F_1| = 2n-3$ and $|F_0| = 2n-4$. This argument is similar to Subcase 3.3.


{\bf Case 4.} $F_0 \neq \emptyset$, $F_2 \neq \emptyset$, and $F_1 = \emptyset$. Then, we have the following two subcases.


{\bf Subcase 4.1.} $1 \leq |F_0| \leq 2n-5$ and $2n-2 \leq |F_2| \leq 4n-8$. Since $\kappa(B_{n-1})=2n-4$ and $F_1 = \emptyset$, $R_0$ and $R_1$ are connected, respectively. For $n \geq 5$, $|S_n|=2^{n-1} > 4n-8$, then $S_n \setminus F_2 \neq \emptyset$. For each node $u$ in $S_n \setminus F_2$, by Definition \ref{Defi-BCDCn}, $u$ has $n-1$ neighbours in $R_1$. Thus, $B[V(R_1) \cup V(R_2)]$ is connected. For each node $x\in V(R_0)$, there are two nodes in $S_n$ which are adjacent to $x$ and for each node $u\in V(R_2)$, $u$ has $n-1$ neighbours in $R_0$. By Definition \ref{Defi-BCDCn},
$|E(R_0,R_2)| \geq (n-1)2^{n-1}-(n-1)(2n-3)-2(2n-4) > 0$ for $n \geq 5$. Thus, there exist two nodes $x \in V(R_0)$, $u \in V(R_2)$ and $(x,u) \in E(B_n)$. Since $R_0$ and $B[V(R_1) \cup V(R_2)]$ are connected, respectively, $B_n-F-V(A)$ is connected.

%{\bf Subcase 4.2.} $2n-4 \leq |F_0| \leq 3n-5$ and $n-2 \leq |F_2| \leq 2n-3$. We can easily verify that $B[V(R_1) \cup V(R_2)]$ is connected. If $R_0$ is connected, similar to Subcase 4.1, $B_n-F-V(A)$ is connected; Suppose that $R_0$ is disconnected. If for each node $x \in V(R_0)$, there exists a neighbour $u$ in $R_2$, then $B_n-F-V(A)$ is connected; By Lemma \ref{BnK1}, there exists only one isolated node $x_1$ and another connected component $L$. Since $F$ is a conditional faulty set, $x_1$ is adjacent to a node $u_1$ in $R_2$. Thus, $B[V(R_1) \cup R_2 \cup \{x_1\}]$ is connected. By Definition \ref{Defi-BCDCn},
%$|E(R_0,R_2)| \geq (n-1)2^{n-1}-(n-1)(2n-3)-2(2n-4) > 0$ for $n \geq 5$. Thus, there exist two nodes $x \in V(R_0\setminus{x_1})$, $u \in V(R_2)$ and $(x,u) \in E(B_n)$. Since $R_0\setminus{x_1}$ and $B[V(R_1) \cup V(R_2) \cup \{x_1\}]$ are connected, respectively, $B_n-F-V(A)$ is connected.
%
%{\bf Subcase 4.3.} $3n-4 \leq |F_0| \leq 4n-8$ and $1 \leq |F_2| \leq n-3$. We can easily verify that $B[V(R_1) \cup V(R_2)]$ is connected. If $R_0$ is connected, similar to Subcase 4.1, $B_n-F-V(A)$ is connected; Suppose that $R_0$ is disconnected. If for each node $x \in V(R_0)$, there exists a neighbour $u$ in $R_2$, then $B_n-F-V(A)$ is connected; Otherwise, for a node $x$ which is not adjacent to a node in $R_2$, by Definition  \ref{Defi-BCDCn} and $F$ is a conditional faulty set, $|E(N_{B^0_{n-1}}(x),S_n)| = 2n-4 > |F_2|$ for $n \geq 5$. Thus, $x$ is either directly connected to a node in $R_2$, or connected to a node in $R_2$ through its neighbors in $R_1$

{\bf Subcase 4.2.} $2n-4 \leq |F_0| \leq 4n-8$ and $1 \leq |F_2| \leq 2n-3$. We can easily verify that $B[V(R_1) \cup V(R_2)]$ is connected. If $R_0$ is connected, similar to Subcase 4.1, $B_n-F-V(A)$ is connected; Suppose that $R_0$ is disconnected. If for each node $x \in V(R_0)$, there exists a neighbour $u$ in $R_2$, then $B_n-F-V(A)$ is connected; Otherwise, since $F$ is a conditional faulty set, there is no an isolated node in $B_n-F$, for each node $x \in V(R_0)$, there exists a node $v$, where $v \in N_{R_0}(x)$. According to the proof at the beginning, $R$ does not contain a $K_2$. Then there exists a node $w$, where $w \in N_{R_0}(x,v)$. If $w \in V(R_2)$, then $B_n-F-V(A)$ is connected; Otherwise, $w \in N_{R_0}(x,v)$. Let $X=N_{R_0}(x,v,w)$, if there exists a node $z \in X$ and $z$ has a neighbour in $V(R_2)$, then $B_n-F-V(A)$ is connected; Otherwise, by Lemma \ref{SnNeighboursNode}, $N_{S_n}(N_{B_n-F}(x,v,w) \cup \{x,v,w\}) > 4n-6 > |F|=4n-7$, a contradiction.

{\bf Case 5.} $F_1 \neq \emptyset$, $F_2 \neq \emptyset$, and $F_0 = \emptyset$. This argument is similar to Subcase 3.3.

{\bf Case 6.} $F_0 \neq \emptyset$, $F_2 \neq \emptyset$, and $F_1 \neq \emptyset$. Then, we have the following four subcases.

{\bf Case 6.1.} $1 \leq |F_0| \leq 2n-5$, $1 \leq |F_1| \leq 2n-5$, and $3 \leq |F_2| \leq 4n-9$. Since $\kappa(B_{n-1})=2n-4$, $R_0$ and $R_1$ are connected, respectively. By Lemma \ref{nodesNeiinBn}, there exists at most a node $u_1 \in R_2$, where $N_{B^0_{n-1}}(u_1) \subset F_1$. For each node $u$ in $R_2 \setminus \{u_1\}$, there are $(n-1)$ nodes in $V(B^1_{n-1})$ which are adjacent to $u$. Thus, $B[\{R_2\setminus{u_1}\} \cup V(R_1)]$ is connected. For $R_0$ and $R_2$, we have $|E(R_0,R_2)| \geq (n-1)2^{n-1}-(n-1)(2n-3)-2(2n-3) > 0$ for $n \geq 5$. Then $B[V(R_0) \cup V(R_2)]$ is connected. Similarly, there exists at most one node $u_2 \in S_n$, where $N_{B^1_{n-1}}(u_2) \subset F_1$ and $u_1 \neq u_2$, and $B[\{R_2 \setminus u_2\} \cap R_1]$ is connected. For $n \geq 5$, $|S_n|=2^{n-1}-(4n-9) > 2$. Thus, $B_n-F-V(A)$ is connected.

%{\bf Case 6.2.} $2n-3 \leq |F_0| \leq 3n-5$, $1 \leq |F_1| \leq 2n-5$, and $1 \leq |F_2| \leq 2n-5$. By Lemma \ref{nodesNeiinBn}, there exists at most a node $u_1 \in R_2$, where $N_{B^1_{n-1}}(u_1) \subset F_1$. For each node $u$ in $R_2 \setminus \{u_1\}$, there are $(n-1)$ nodes in $V(B^1_{n-1})$ which are adjacent to $u$. Thus, $B[\{R_2\setminus{u_1}\} \cup V(R_1)]$ is connected. Since $F$ is a conditional faulty set, there is no an isolated node in $B_n-F$. There exists a node $x_1$ in $V(R_0)$, where $(x_1,u_1)\in E(B_n)$. Since $R$ does not contain a $K_2$, $x_1$ is adjacent to a node $x_2$ in $R_0$ or a node $u_2$ in $R_2$ $(u_1 \neq u_2)$. If $R_0$ is connected, similar to Subcase 6.1, $B_n-F-V(A)$ is connected; Suppose that $R_0$ is disconnected. If for each node $x \in V(R_0)$, there exists a neighbour $u$ in $R_2$, then $B_n-F-V(A)$ is connected; By Lemma \ref{BnK1}, there exists only one isolated node $x_1$ and another connected component $L$. Since $F$ is a conditional faulty set, $x_1$ is adjacent to a node $u_1$ in $R_2$. Thus, $B[V(R_1) \cup R_2\setminus\{u_1\} \cup \{x_1\}]$ is connected. By Definition \ref{Defi-BCDCn},
%$|E(R_0,R_2)| \geq (n-1)2^{n-1}-(n-1)(2n-3)-2(2n-4) > 0$ for $n \geq 5$. Thus, there exist two nodes $x \in V(R_0\setminus{x_1})$, $u \in V(R_2)$ and $(x,u) \in E(B_n)$. Since $R_0\setminus{x_1} \cup {u_1}$ and $B[V(R_1) \cup R_2\setminus\{u_1\} \cup \{x_1\}]$ are connected, respectively, $B_n-F-V(A)$ is connected.
%
%{\bf Case 6.3.} $3n-4 \leq |F_0| \leq 4n-9$, $1 \leq |F_1| \leq n-4$, and $1 \leq |F_2| \leq n-4$. We can easily verify that $B[V(R_1) \cup V(R_2)]$ is connected. If $R_0$ is connected, similar to Subcase 6.1, $B_n-F-V(A)$ is connected; Suppose that $R_0$ is disconnected. If for each node $x \in V(R_0)$, there exists a neighbour $u$ in $R_2$, then $B_n-F-V(A)$ is connected; Otherwise, for a node $x$ which is not adjacent to a node in $R_2$, by Definition  \ref{Defi-BCDCn} and $F$ is a conditional faulty set, $|E(N_{B^0_{n-1}}(x),S_n)| = 2n-4 > |F_2|$ for $n \geq 5$. Thus, $x$ is either directly connected to a node in $R_2$, or connected to a node in $R_2$ through its neighbors in $R_1$. Thus, $B_n-F-V(A)$ is connected.
%
%{\bf Case 6.4.} $2n-3 \leq |F_1| \leq 4n-9$, $1 \leq |F_0| \leq 2n-5$, and $1 \leq |F_2| \leq 2n-5$. This argument is similar to Subcase 6.2.

{\bf Subcase 6.2.} $2n-4 \leq |F_0| \leq 4n-9$,  and $1 \leq |F_2| \leq 2n-5$ and $1 \leq |F_2| \leq 2n-5$. We can easily verify that $B[V(R_1) \cup V(R_2)]$ is connected. If $R_0$ is connected, similar to Subcase 4.1, $B_n-F-V(A)$ is connected; Suppose that $R_0$ is disconnected. If for each node $x \in V(R_0)$, there exists a neighbour $u$ in $R_2$, then $B_n-F-V(A)$ is connected; Otherwise, since $F$ is a conditional faulty set, there is no an isolated node in $B_n-F$, for each node $x \in V(R_0)$, there exists a node $v$, where $v \in N_{R_0}(x)$. According to the proof at the beginning, $R$ does not contain a $K_2$. Then there exists a node $w$, where $w \in N_{R_0}(x,v)$. If $w \in V(R_2)$, then $B_n-F-V(A)$ is connected; Otherwise, $w \in N_{R_0}(x,v)$. Let $X=N_{R_0}(x,v,w)$, if there exists a node $z \in X$ and $z$ has a neighbour in $V(R_2)$, then $B_n-F-V(A)$ is connected; Otherwise, by Lemma \ref{SnNeighboursNode}, $N_{S_n}(N_{B_n-F}(x,v,w) \cup \{x,v,w\}) > 4n-6 > |F|=4n-7$, a contradiction.

%{\bf Case 6.5.} $|F_0| = 2n-4$, $|F_1| = 2n-4$, and $|F_2|=1$. If $R_0$ is connected, by Lemma \ref{nodesNeiinBn}, there exists at most a node $u_1 \in R_2$, where $N_{B^0_{n-1}}(u_1) \subset F_1$. For each node $u$ in $R_2 \setminus \{u_1\}$, there are $(n-1)$ nodes in $V(B^0_{n-1})$ which are adjacent to $u$. Thus, $B[\{R_2\setminus{u_1}\} \cup V(R_0)]$ is connected. Since $F$ is a conditional faulty set, there is no an isolated node in $B_n-F$. There exists a node $y_1$ in $V(R_0)$, where $(y_1,u_1)\in E(B_n)$.
%Since $R$ does not contain a $K_2$, $y_1$ is adjacent to a node $y_2$ in $R_0$ or a node $u_2$ in $R_2$ $(u_1 \neq u_2)$.
%For each node $y$ in $R_0$, there are two neighbours in $S_n$, Thus, $B_n-F-V(A)$ is connected; If $R_0$ and $R_1$ are disconnected, respectively, by Lemma \ref{BnK1}, there exists only one isolated node $y_1$ and another connected component $L$. By Definition \ref{Defi-BCDCn}, $y_1$ has two neighbours $u_1$ and $u_2$ in $S_n$. By Lemma \ref{nodesNeiinBn}, min$(|N_{B_{n-1}^1}(\{u_1,u_2\})|)=2n-3 > N_{B_{n-1}^1(y_1)}=|F_1|=2n-4$, so there exists a node $y_2$ in $L$ which is adjacent to $u_1$ or $u_2$. Without loss of generality, we set $(y_2,u_1)\in E(B_n)$. Since $L$ is connected and $B[\{y_1,u_1,u_2\}]$ is connected, $B[S_n \cap V(R_1)]$ is connected. For each node $x$ in $V(R_0)$, Similarly, $B[S_n \cap V(R_0)]$ is connected. Thus, $B_n-F-V(A)$ is connected.

The above arguments indicate that if there is no $K_2$ in $B_n-F$, then $B_n-F$ is connected. Therefore, the Lemma holds.
\end{proof}

\begin{lem} \label{xBelongstoL} \rm
Let $F_1$ and $F_2$ be two different conditional faulty node sets in $B_n$, where $F_1 \leq 4n-6$ and $F_2 \leq 4n-6$, and let $L$ be the largest component in $B_n-(F_1 \cap F_2)$, then for each node $x \in F_1 \bigtriangleup F_2$, $x \in L$.
\end{lem}

\begin{proof}
Without loss of generality, assume that $x \in F_2-F_1$. Since $F_2$ is a conditional fault node set in $B_n$, $x$ has a neighbour $y \in V(B_n)-F_2-x$. Suppose that $x \notin L$, then $B_n-(F_1 \cap F_2)$ is disconnected. Since $F_1 \bigtriangleup F_2 \leq 4n-7$, by Lemma \ref{BnK2}, $\{x,y\}$ belong to the smaller component $K_2$. Then, all the
neighbors of $y$ belong to $F_1$, which contradicts the fact that
$F_2$ is a conditional faulty set.
\end{proof}

\begin{lem} \label{conditionalDiagnosability<=} \rm
$t_c(B_n) \leq 4n-6$, for $n \leq 6$.
\end{lem}

\begin{proof}
If $F_1 \subset F_2$ or $F_2 \subset F_1$, by Lemmas \ref{connectivity} and \ref{F1belongstoF2}, $F_1$ and $F_1$ are distinguishable. Then we consider that $F_1-F_2 \neq \emptyset$ and $F_2-F_1 \neq \emptyset$. Let $S=F_1 \cap F_2$ with $|S|\leq 4n-7$, and $L$ be the maximum component of $B_n-S$. From Lemma \ref{xBelongstoL}, we know that for each node $u \in F_1 \bigtriangleup F_2$, $u \in L$. Next, we will prove that there exists a node $x \in L \setminus \{F_1 \cup F_2\}$, which has no neighbour in $S$. Then we will complete our proof by $u$ and $x$.

By Lemmas \ref{connectivity} and \ref{BnK2}, we have $N_{B_n}(S) \leq (2n-2)|S|$ and $|V(L)| \geq V(B_n)-|S|-2=n2^{n-1}-|S|-2$. Then,
$|V(L)|-|F_1 \bigtriangleup F_2|-(2n-2)|S| \geq n2^{n-1}-|S|-2-(2(4n-6)-2|S|)-(2n-2)|S| \geq n2^{n-1}-8n+10-(2n-3)|S| \geq n2^{n-1}-8n+10-(2n-3)(4n-7) \geq 6 \times 2^{6-1}-8\times6+10-(2\times6-3)(4\times6-7)=1$.

For each node $x$ in $|V(L)|-|F_1 \bigtriangleup F_2|-(2n-2)|S|$, we have the following two cases:

{\bf Case 1.} $N_{B_n}(x) \cap (F_1 \bigtriangleup F_2) \neq \emptyset$. If $x$ has two or more neighbours in $F_1-F_2$ $(F_2-F_1)$, condition 1 (condition 2) of Lemma \ref{distinguishable} is satisfied; If $x$ has only one neighbour in $F_1-F_2$ $(F_2-F_1)$, since $x$ has no neighbours in $S$, there exists a neighbour which belongs to $V(G) \setminus (F_1 \cup F_2)$. Thus, condition 3 of Lemma \ref{distinguishable} is satisfied.

{\bf Case 2.} $N_{B_n}(x) \cap (F_1 \bigtriangleup F_2) = \emptyset$. Since for each node $u \in F_1 \bigtriangleup F_2$, $u \in L$ and $L$ is connected, there exists a path $P=\langle x(=x_0),x_1,\ldots,x_j,u \rangle$ where $|P| \geq 2$, and $u \in F_1 \bigtriangleup F_2$, and $x_j \in V(L)-F_1 \bigtriangleup F_2$ with $j \geq 1$. Then, $x_j$ has a neighbour in $F_1 \bigtriangleup F_2$ $(u)$ and a neighbour $x_{j-1}$ in $L \setminus (F_1 \cup F_2)$. Thus, condition 3 of Lemma \ref{distinguishable} is satisfied.
\end{proof}

\section{The Conditional Diagnosability of BCDC under the PMC Model}\label{PMC}

\begin{lem} \label{C4} \rm
Let $F$ be a conditional faulty set of $B_n$ with $n \geq 6$. $B_n-F$ is disconnected and there exists a component $P$ with $N_{P}(u) \geq 2$ for $u \in V(P)$. Then one of two following conditions holds: 1) $|F| \geq 5n-8$; 2) $|V(P)| \geq 4n-5$.
\end{lem}

\begin{proof}
Let $F_0=F \cap V(B^0_{n-1})$, $F_1=F \cap V(B^1_{n-1})$, $F_2 = F \cap V(S_n)$. According to the distribution of $P$, we have the following cases:

{\bf Case 1.} $V(P) \subset V(B^0_{n-1})$ or $V(P) \subset V(B^1_{n-1})$.

Without loss of generality, suppose that $V(P) \subseteq V(B^0_{n-1})$. Then, either $V(B^0_{n-1})=V(P) \cup V(F_0)$ or $B^0_{n-1}-F_0$ is disconnected. If $V(B^0_{n-1})=V(P)$, since $|F| \leq 6n-8$, $V(P)= V(B^0_{n-1})-(6n-8) \geq 4n-4$. If $B^0_{n-1}-F_0$ is disconnected and $|P| \geq 4n-4$, then condition 2 holds. Otherwise, we consider $|P| \leq 4n-5$. We choose a node set $C$ in $V(P)$ with $|V(C)|=4$. Since $V(C) \subseteq V(P) $, $N_{B^0_{n-1}}(C) \subseteq F_0 \cup (V(P)-V(C))$. Then we have $ N_{B^0_{n-1}}(C)- (V(P)-V(C))\subseteq F_0$. Thus, $|F_0| \geq |N_{B^0_{n-1}}(C)|- |V(P)-V(C)|=|N_{B^0_{n-1}}(C)|- |V(P)|+|V(C)|=|N_{B^0_{n-1}}(C)|- |V(P)|+4$. On the other hand, we have $|F_2| \geq |N_{S_n}(P)|$,
\end{proof}

\section{Conclusions}\label{conclusion}

conclusions

\section*{Acknowledgement}
acknowledgement

\setcounter{tocdepth}{1}
\begin{thebibliography}{20}
%1
\bibitem{wang2018bcdc}
X. Wang, J. X. Fan, C.-K. Lin, and J. Y. Zhou, ``BCDC: a high-performance, server-centric data center network,'' Journal of Computer Science and Technology, vol. 33, no. 2, pp. 400-416, 2018.
%2
\bibitem{Bouabdallah}
A. Bouabdallah, M. C. Heydemann, J. Opatrny, D. Sotteau, Embedding complete binary trees into star and pancake graphs, Theory of Computing Systems, 31(3) (1998) 279-305.


\end{thebibliography}


\end{document}
