\documentclass[12pt]{article}
\usepackage{amsthm}
\usepackage{amsmath}
\usepackage{geometry}
\usepackage{graphicx}
\usepackage{epstopdf}
\usepackage{color}
\usepackage{cite}
%��ͼ���ţ���Ҫ���
\usepackage{subfigure}
\newtheorem{dfn}{Definition}
\newtheorem{thm}{Theorem}
\newtheorem{lem}{Lemma}
\newtheorem*{prof}{Proof}
\newtheorem{prop}{Property}
\geometry{left=2.0cm,right=2.0cm,top=2.5cm,bottom=2.5cm}
%Ҫ����u_i,�ϻ���(u)_i
%N(u)ҲҪ����һ��
%V(u)Ҫ����
%����һ���ṹ����form,construct,����constitute

\newcommand{\comment}[2]{\textcolor{red}{#1:#2}}

\begin{document}
%\title{\textbf{Structure connectivity and substructure connectivity of  the Augmented Cube}}

\begin{center}
\Large\bf The 3-extra connectivity of Augmented cubes\\
\vspace{0.5cm}
  \large
     Shuangxiang Kan$^1$, Jianxi Fan$^{1*}$, Baolei Cheng$^1$, Xi Wang$^2$, Jingya Zhou$^1$   \\
    $^1$School of Computer Science and Technology, Soochow University, Suzhou 215006, China\\
    $^2$School of Software and Services Outsourcing, Suzhou Institute of Industrial Technology, Suzhou 215004, China\\
    $^*$Corresponding author: jxfan@suda.edu.cn\\
\end{center}

%\author{\textbf{shuangxiang kan}}
\begin{center}
\textbf{Abstract}
\end{center}

\textbf{Keywords:} structure connectivity; substructure connectivity; fault-tolerant ability; augmented cube; interconnection network
\section{Introduction}
Given a graph $G$ and a non-negative integer $g$, if there is a set of vertices in the graph $G$ such that the graph $G$ is disconnected after the vertex set is deleted and the number of vertices of each component is greater than $g$, then we call it a vertex cut. The minimum cardinality of all vertex cuts is referred to as the $g$-$extra$ $connectivity$ of graph $G$, denoted by $\kappa_g(G)$. $g$-extra connectivity is a generalization of the connectivity.
\section{$H$-structure connectivity and $H$-substructure connectivity}
\begin{thm}
\label{connectivity}
$\kappa(AQ_1)=1$, $\kappa(AQ_2)=3$, $\kappa(AQ_3)=4$, and for $n\geq 4$, $\kappa(AQ_n)=2n-1$.
\end{thm}

$\kappa_2(G)=3$

\begin{lem}
\label{2nodes}
Let $x$, $y$ be two nodes in $AQ_n$, where $n\geq3$. Then
$N_{AQ_n}(\{x,y\})\geq4n-8$ holds.
\end{lem}


\begin{lem}
\label{3nodes}
Let $x$, $y$, $z$ be three nodes in $AQ_n$, where $n\geq3$. Then
$N_{AQ_n}(\{x,y,z\})\geq6n-17$ holds.
\end{lem}

\begin{lem}
\label{2extra}
$\kappa_1(AQ_n)=4n-8$ for $n\geq6$.
\end{lem}

\begin{lem}
\label{3extra<=}
$\kappa_2(AQ_n)<6n-17$ for $n\geq6$.
\end{lem}

\begin{proof}
\end{proof}

\begin{lem}
\label{3extra<=}
$\kappa_2(AQ_n)=6n-17$ for $n\geq6$.
\end{lem}

\begin{proof}
Let $S$ be an arbitrary set of vertices in $AQ_n$ such that $|S| \leq 6n-18$ and there are no isolated vertices and $K_2$s in $AQ_n-S$. We will prove that $AQ_n-S$ is connected.
Note that $AQ_n=L \oplus R$ where $L \cong AQ^0_{n-1}$ and $R \cong AQ^1_{n-1}$. For convenience, let $S_L=S \cap L$ and $S_R=S \cap R$. Without loss of generality, we may suppose that $|S_L| \geq |S_R|$. Then $|S_R|\leq 3n-9$.

Then we have the following two cases:\\
\textbf{Case 1}: $S_R$ is connected.

We need to prove that any vertex in $L-S_L$ is connected via a path to a vertex in $R-S_R$. Let $u$ be any vertex in $L-S_L$, if its two neighbours
$u_n$ and $\overline{u}_n$ don't belong to $R-S_R$. Then we are done. Otherwise, we have $\{u_n,\overline{u}_n\} \subset R-S_R$.

Since $AQ_n-S$ does not contain isolated vertices, then there exists a neighbour $v$ of $u$, if $\{v_n,\overline{v}_n\}\not\subset R-S_R$, we are done. Otherwise, we have $\{v_n,\overline{v}_n\} \subset R-S_R$.

Since $AQ_n-S$ does not contain $K_2$s, then there exists a neighbour $w$ of $u$ and $v$, if $\{w_n,\overline{w}_n\}\not\subset R-S_R$, we are done. Otherwise, we have $\{w_n,\overline{w}_n\} \subset R-S_R$. If $\{u,v,w\}$ is an isolated $K_3$, then by Lemma \ref{3nodes}, we have $N_{AQ_n}(\{u,v,w\})=6n-17$, a contradiction. So we assume $\{u,v,w\}$ is not an isolated $K_3$. Since $N_{AQ_n}(\{u,v,w\})=6n-17$ and $|S_R|\leq 3n-9$, and for any vertex $m$ in $L$, $m$ can connect $L$ via $m_n$ and $\overline{m}_n$. Thus, for any vertex $u$ in $L-S_L$, we can get a path $\{u,y,z\}$ where $y \in N_{L}(u,v,w)$ and $z \in \{y_n,\overline{y}_n\}$ and $z \in R-S_R$;
%if there exists a neighbour $y$ of $u$, $v$ and $w$, then\\

\textbf{Case 2}: $S_R$ is disconnected.

And there is an isolated vertex $x$ in $R-S_R$ since $|S_R|\leq 3n-9$ and $\kappa(AQ_{n-1})=2n-3$ and $\kappa_1(AQ_{n-1})=4n-12$.

Since $AQ_n-S$ does not contain isolated vertices and $K_2$s, then there exists a $K_2=\{v,w\}$ that $\{v,w\} \subset L-S_L$ and $u$ connects $\{v,w\}$. Since $u$ is an isolated vertex in $R-S_R$, then $S_L<=6n-18-(2n-3)=4n-15$. And by Lemma \ref{2nodes}, $N_{AQ^0_n}(v,w)>=4n-12$, there there is a vertex $y \notin L-S_L$ and $y \in N_{L-S_L}\{v,w\}$. Then by Lemma \ref{3nodes}, we have $N_{L-S_L}\{v,w,y\}>=6n-23 > 4n-15$ for $n \geq 6$. Then according to the definition of $AQ_n$ and similar to Case 1,

\end{proof}



\section{Conclusions}
conclusion

\section*{Acknowledgment}
acknowledgement
\setcounter{tocdepth}{1}
\begin{thebibliography}{20}

\bibitem{LvY2017}
Y. Lv, C.-K. Lin, J. Fan, Hamiltonian cycle and path embeddings in $k$-ary $n$-cubes based on structure faults, The Computer Journal 60(2) (2017) 159-179.

\end{thebibliography}

\end{document}
